% Presentación del tema 
\section{Teoría de átomos en moléculas}
  % Investigar si aditividad es correcto en español
  La capacidad de átomos individuales o grupos funcionales de exhibir
  propiedades transferibles de una molécula a otra ha desempeñado un
  papel crucial en el desarrollo de la química. En algunas series de
  moléculas, la variación es tan mínima, que se pueden crear esquemas
  de aditividad para propiedades específicas.
  
  Es aquí donde se origina el núcleo de la teoría cuántica de átomos 
  en moléculas (\textsc{qtaim}, por sus siglas en inglés), desarrollada
  por Richard F. W. Bader y colaboradores. Esta teoría busca comprender
  la base electrónica de esta transferencia empírica postulando la 
  existencia de átomos en moléculas.
  %
\subsection{La densidad electrónica $\rho(\vec{r})$}
  Cuestionar la existencia de átomos en moléculas es esencialmente cuestionar si 
	la función de onda $\Psi(\vec{X}, t)$, que contiene toda la información del sistema 
	\cite{cohen1}, puede proporcionar una forma única de dividir la molécula en subsistemas, 
	permitiendo definir observables,sus valores promedio y las ecuaciones de movimiento 
  correspondientes a cada subsistema. Al desarrollarse la física del átomo en la molécula 
  dentro del marco de la mecánica cuántica, se pudo responder a estas interrogantes 
  \footnote{Para un análisis más profundo, veáse los capítulos 5 y 7 de \cite{badero}}.
		
	De hecho, Bader y Beddall \cite{bader1972} realizaron un descubrimiento de gran relevancia: los 
  subsistemas cuánticos se comportan como sistemas abiertos en el espacio real y sus fronteras 
  pueden ser determinadas mediante una propiedad específica de la densidad electrónica. Así, aunque 
  esta teoría tiene sus raíces en la mecánica cuántica, su formulación y aplicación se basan en el 
  concepto de densidad. 
		
	En el caso de un sistema compuesto por $N$ partículas, la función de onda se vuelve una función
  de $4N$ variables, $\Psi(\vec{X_1}, \vec{X_2}, \cdots, \vec{X_n})$, donde $\vec{X_i} = 
  \{x_i,y_i,z_i,\sigma_i\}$ representa las coordenadas y el espín de la partícula $i$. Por lo 
  tanto, el análisis de $\Psi$ se vuelve complicado, especialmente para sistemas más grandes. 
  En su lugar, puede considerarse a $\rho(\vec{r})$ :
	%
  \begin{equation*}
			\rho(\vec{r}) = \rho(x,y,z) \propto |\Psi|^2,
	\end{equation*}
	%
  una función de tres variables que se puede medir mediante difracción de rayos X. La densidad 
	electrónica proporciona una medida de la probabilidad de encontrar un electrón en las proximidades del 
	punto $\vec{r}$. Aunque se ha logrado un conocimiento limitado sobre la densidad electrónica, la solución 
		exacta de la ecuación de Schrödinger electrónica para el átomo de hidrógeno nos revela una distribución radial
		exponencial con un decaimiento monótono a grandes distancias \cite{decayalhrichs} \cite{hoffmannh2}. Cerca del núcleo, 
		la condición de cúspide de Kato \cite{kato57} establece que la derivada de la densidad está relacionada con el número
	   atómico $Z$ mediante la expresión:
	  \begin{equation}
			Z = - \frac{a_0}{2\rho(\vec{r})}
			\frac{d}{d\vec{r}}
			\rho(\vec{r})
			\Biggr|_{\substack{\vec{r} \rightarrow \vec{R}}}
		\end{equation}
	donde $a_0$ representa el radio atómico de Bohr y $\vec{R}$ corresponde a las coordenadas del núcleo. En la \Cref{fig:LiHnube}, 
	se muestra una representación de la 	densidad electrónica cuando $x=0$ para la molécula de \ch{LiH}. El conocimiento de 
	la densidad electrónica nos permite comprender diversas propiedades del sistema, como su actividad química, mecanismos de reacción, 
	entre otras.
	%
	\begin{figure} [h!]
		\centering
		\includegraphics[scale=0.6]{LiHnube.png}
		\caption{La nube de $\rho(x,y,z)$ del \ch{LiH} (\textsc{hf/sto-3g})}
		\label{fig:LiHnube}
	\end{figure}
	%
	\subsection{La superficie de flujo cero}
	La \textsc{qtaim} ofrece un enfoque riguroso para la partición de una molécula en sus componentes atómicos 
	mediante el análisis del gradiente de la densidad electrónica:
	%	
	\begin{equation}
		\nabla \rho(x,y,z) = \frac{\partial \rho(x,y,z)}{\partial x} \vec{i} +
		\frac{\partial \rho(x,y,z)}{\partial y} \vec{j} +
		\frac{\partial \rho(x,y,z)}{\partial z} \vec{k}
	\end{equation}
	%
	Aquí, $\vec{i}$, $\vec{j}$ y $\vec{k}$ son vectores unitarios en los ejes de coordenadas correspondientes. 
	El vector gradiente $\nabla \rho(x,y,z)$ indica la dirección de mayor crecimiento de la función $\rho(x,y,z)$, mientras que el 
	negativo del gradiente apunta a la dirección de mayor disminución de $\rho$. Siguiendo las trayectorias definidas por $\rho(x,y,z)$, 
	conocidas como trayectorias del gradiente del campo gradiente, se pueden identificar de manera precisa las superficies atómicas, 
	también conocidas como superficies de " flujo cero''. Estas superficies son utilizadas por Bader y sus colaboradores para definir los átomos 
	en una molécula. En consecuencia, desempeñan un papel fundamental en la construcción de las unidades básicas de la 
	\textsc{qtaim}, las cuencas atómicas $\Omega$. Además, se utilizan ampliamente en la evaluación de propiedades electrónicas como la
	 carga atómica (conocida como carga de Bader), los momentos dipolares, las energías atómicas y otras propiedades relevantes.
	
	En la  \Cref{fig:ch2fig}, se ilustra cómo las trayectorias del campo gradiente de la función de densidad electrónica, $\rho(x,y,z)$, 
	dividen la molécula de \ch{CH2^-} en sus componentes atómicos de manera precisa y sin ambigüedades.
	%
	\begin{figure}[!ht] 
		\centering
		\includegraphics[scale=0.15]{ch2.jpg}
	%\label{fig:bolzano}
	\caption{El mapa de contorno de $\rho(\vec{r})$ (rojo) de la molécula 
	de \ch{CH2-} con tres regiones atómicas (azul y verde) definidas por las 
	trayectorias de $\nabla\rho(\vec{r})$.}
	\label{fig:ch2fig}
	\end{figure}	
	
	\subsection{La malla de Lebedev-Euler-McLaurin}
	La determinación de las propiedades atómicas implica la evaluación de la integral
	sobre la cuenca $\Omega$ 
	%
	\begin{equation*}
		\int_\Omega\Psi{\cal O}\Psi\ \df{d}\vec{r}.
	\end{equation*}	
	%
	Por ejemplo cuando el operador $\cal O$ es la matriz unidad y $\Omega$ 
	corresponde al átomo $A$, la integral relevante es:
	%
	\begin{equation*}
		\int_{\Omega_A}\rho(\vec{r})\ \df{d}\vec{r} = q_A
	\end{equation*}
	%
	donde se obtiene la llamada carga atómica de Bader del átomo $A$. 

	\section{Objetivos y Justificación}
    Hasta el momento, los algoritmos utilizados para construir las superficies de flujo cero se basan en el 
    número y la distribución uniforme de trayectorias 	\cite{biegler1981,popelier96}, así como en la evaluación 
    analítica de $\nabla\rho(x,y,z)$. Sin embargo, este enfoque no resulta eficiente, especialmente 
    cuando se trata de la región de enlace, que es de principal interés. Además, en el caso de interacciones débiles, 
    donde tanto $\rho(\vec{r})$ como su gradiente son valores pequeños, este método puede presentar inestabilidades
     numéricas. La construcción de la superficie atómica se convierte así en un punto crítico en los cálculos 
     de la (\textsc{qtaim})

	Nuestro objetivo principal es desarrollar una  implementación en el lenguaje de programación C++
	para construir las superficies atómicas  $\Omega$ de Bader.  Además, buscamos evaluar integrales 
	relevantes usando el esquema de malla de Lebedev-Euler-McLaurin. Este algoritmo aprovechará la 
	paralelización a través de las directivas de \texttt{openMP} \cite{openmp}.
%
